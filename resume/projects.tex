%-------------------------------------------------------------------------------
%	SECTION TITLE
%-------------------------------------------------------------------------------
\cvsection{Programming Projects}


%-------------------------------------------------------------------------------
%	CONTENT
%-------------------------------------------------------------------------------
\begin{cventries}
    \cventry
    {\href{https://github.com/HDonovan96/PacmanAI}{https://github.com/HDonovan96/PacmanAI}}
    {Pacman AI}
    {Unity 2019.4.1}
    {December 2020 - January 2021}
    {
        \begin{cvitems}
            \item 
        \end{cvitems}
    }

    \cventry
    {\href{https://github.com/HDonovan96/Snake}{https://github.com/HDonovan96/Snake}}
    {Snake}
    {C++ OpenGL}
    {July 2019}
    {
        \begin{cvitems}
            \item Developed and implemented in OpenGL for C++
            \item Programmed a game of snake including collision detection and random 'apple' placement
            \item Displays players score upon death without the use of text as a challenge to see if the information could still be conveyed clearly
            \item Set up initial conditions so the game could be restarted with a single key press
            % \item Scaled movement speed with the length of the snake to increase the difficulty as the player does better 
        \end{cvitems}  
    }

    % \cventry
    % {\href{https://github.com/HDonovan96/Iterative_Rocket_Glider_Design}{Rocket Glider Github Link}}
    % {Rocket Glider Analysis}
    % {C++}
    % {January 2019 - March 2019}
    % {
    %     \begin{cvitems}
    %         \item Used an iterative method to find the best dimensions for a rocket powered glider from know boundaries
    %         \item V1.1 was optimisation and reduced the run time by 60\%
    %     \end{cvitems}
    % }
    
    % \cventry
    % {\href{https://github.com/HDonovan96/Programming_Nursery/tree/master/KOS}{Auto-Pilot Github Link}}
    % {Auto-Pilot Script}
    % {kOS}
    % {January 2019 - March 2019}
    % {
    %     \begin{cvitems}
    %         \item Used OOP to create a script to take a rocket from the launch pad to a sub-orbital path
    %         \item Improved the script to automatically circularise the orbit once a sub-orbital trajectory is achieved
    %     \end{cvitems}
    % }
\end{cventries}